\chapter{Introduction}
\citep{Kuijken_2008}, \cite{Refregier_2003_1}, \cite{Refregier_2003_2} \cite{Kuijken_2015}

% \begin{equation}
%     L_{lmn}(a,b,c) = \tfrac{1}{\sqrt{\pi}}\int dx\; e^{-x^2}H_l(ax)H_m(bx)H_n(cx)
% \end{equation}

% Using the explicit expression of the Hermite polynomial

% \begin{align}
%     H_n(ax)&=\sum_{m=0}^{\lfloor\frac{n}{2}\rfloor}\frac{n!(-1)^m}{m!(n-2m)!}(2ax)^{n-2m}\\&=\sum_{m=0}^{\lfloor\frac{n}{2}\rfloor}h_{m,n}(a)x^{n-2m}
% \end{align}

% with 

% \begin{equation}
%     h_{m,n}(a)=\frac{n!(-1)^m(2a)^{n-2m}}{m!(n-2m)!}
% \end{equation}

% we find

% \begin{align}
%     L_{lmn}(a,b,c) &= \tfrac{1}{\sqrt{\pi}}\int dx\; e^{-x^2}\sum_{i=0}^{\lfloor\frac{l}{2}\rfloor}h_{i,l}(a)x^{l-2i}\sum_{j=0}^{\lfloor\frac{m}{2}\rfloor}h_{j,m}(b)x^{m-2j}\sum_{k=0}^{\lfloor\frac{n}{2}\rfloor}h_{k,n}(c)x^{n-2k}\\
%      &= \tfrac{1}{\sqrt{\pi}}\int dx\; e^{-x^2}\sum_{i=0}^{\lfloor\frac{l}{2}\rfloor}\sum_{j=0}^{\lfloor\frac{m}{2}\rfloor}\sum_{k=0}^{\lfloor\frac{n}{2}\rfloor}h_{i,l}(a)h_{j,m}(b)h_{k,n}(c)x^{l+m+n-2(i+j+k)}\\
%      &= \tfrac{1}{\sqrt{\pi}}\sum_{i=0}^{\lfloor\frac{l}{2}\rfloor}\sum_{j=0}^{\lfloor\frac{m}{2}\rfloor}\sum_{k=0}^{\lfloor\frac{n}{2}\rfloor}h_{i,l}(a)h_{j,m}(b)h_{k,n}(c)\int dx\; e^{-x^2}x^{l+m+n-2(i+j+k)}
% \end{align}

% Using the fact that $L_{lmn}$ is only non-zero when $l+m+n$ is even, we can use

% \begin{equation}
%     \int dx\;x^{2n}e^{-x^2}=\sqrt{\pi}\frac{(2n)!}{n!}\left(\frac{1}{2}\right)^{2n}
% \end{equation}

% to evaluate the integral.

% % \begin{align}
% %     L_{lmn}(a,b,c) &= \sum_{i=0}^{\lfloor\frac{l}{2}\rfloor}\sum_{j=0}^{\lfloor\frac{m}{2}\rfloor}\sum_{k=0}^{\lfloor\frac{n}{2}\rfloor}h_{i,l}(a)h_{j,m}(b)h_{k,n}(c)\frac{(l+m+n-2(i+j+k))!}{((l+m+n)/2-(i+j+k)!}\left(\frac{1}{2}\right)^{l+m+n-2(i+j+k)}
% % \end{align}

% \begin{align}
% L_{lmn}(a,b,c) &= 
% \sum_{i=0}^{\lfloor\frac{l}{2}\rfloor}
% \sum_{j=0}^{\lfloor\frac{m}{2}\rfloor}
% \sum_{k=0}^{\lfloor\frac{n}{2}\rfloor}
% h_{i,l}(a)\, h_{j,m}(b)\, h_{k,n}(c) \\
% &\quad\;\, \times 
% \frac{(l+m+n-2(i+j+k))!}
%      { \left( \frac{l+m+n}{2} - (i+j+k) \right)! }
% \left( \frac{1}{2} \right)^{l+m+n-2(i+j+k)}
% \end{align}

Aperture photometry can be written like

\begin{equation}
    F_A = \int\int dxdyO(x,y)W_A(x,y)
\end{equation}

where $O(x,y)$ is the observed data and $W_A(x,y)$ some weight function. We know that $O(x,y)$ is the convolution of the intrinsic image with the PSF

\begin{equation}
    O(x,y) = \int\int dx'dy'P(x-x', y-y')I(x',y')=P\otimes I
\end{equation}

We can use this to rewrite

\begin{equation}
    F_A = \int\int dxdy\int\int dx'dy'P(x-x', y-y')I(x',y')W_A(x,y)
\end{equation}

a change of coordinates, then gives

\begin{equation}
    F_A = \int\int dxdyI(x,y)\tilde{W}_A^P(x,y)
\end{equation}

where 

\begin{equation}
    \tilde{W}_A^P(x,y) = \int\int dx'dy'P(x'-x, y'-y)W_A(x',y')=\bar{P}\otimes W_A
\end{equation}

where $\bar{P}(x,y)=P(-x,-y)$.

The algorithm then works as follows:
\begin{enumerate}
    \item Choose an aperture function $\tilde{W}_A$, which is going to be applied to all bands
    \item Calculate the post-seeing aperture $\tilde{W}_A^i$ by deconvolving the aperture function $\tilde{W}_A$ by the reflected PSF $\bar{P}_i$, of band $i$, $W^i_A=\tilde{W}_A\otimes^{-1}\bar{P}_i$
    \item Calculate the aperture flux for each band using the deconvolved aperture function $W_A^i$.
\end{enumerate}
